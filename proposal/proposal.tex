\documentclass[12pt]{article}
\usepackage{graphicx}
\usepackage{amsmath}
\usepackage{geometry}

\geometry{top=1in, bottom=1in, left=1in, right=1in}

\title{Project Proposal: Fashion Image Classification System with CNN}
\author{}
\date{}

\begin{document}

\maketitle

\section*{Topic:}
Fashion Image Classification System with Convolutional Neural Network (CNN)

\section*{Aim:}
The aim of this project is to develop a machine learning model that classifies images of fashion items into broad categories (such as Apparel, Footwear, Accessories, etc.) with a high degree of accuracy.

\section*{Dataset:}
We will use the Fashion Product Images Dataset from Kaggle: \\
\texttt{Dataset Link} \\
The dataset contains images of fashion products along with their respective categories, such as apparel, footwear, and accessories. The data is organized in two primary components:
\begin{itemize}
    \item \textbf{Images:} Images of fashion items located in the \texttt{images/} directory, with file names corresponding to the product.
    \item \textbf{Labels:} A CSV file \texttt{styles.csv} containing labels for each product.
\end{itemize}

\section*{Model:}
Convolutional Neural Network (CNN) will be used for this project due to its effectiveness in image classification tasks. We will also consider leveraging pretrained models such as ResNet or EfficientNet to improve classification accuracy and reduce training time.

\subsection*{Approach:}
\textbf{Preprocessing:}
\begin{itemize}
    \item Resize all images to a uniform size for input into the model.
    \item Normalize pixel values to ensure consistent scaling across images.
    \item Apply data augmentation techniques (e.g., rotations, flips, color jittering) to artificially increase the dataset size and improve model robustness.
\end{itemize}

\textbf{Model Architecture:}
\begin{itemize}
    \item Start with a simple CNN architecture (Convolutional layers + pooling + fully connected layers).
    \item Optionally, employ transfer learning using pretrained models like ResNet or EfficientNet for feature extraction, followed by fine-tuning on the fashion dataset.
\end{itemize}

\textbf{Training and Evaluation:}
\begin{itemize}
    \item Train the CNN model using the labeled dataset (\texttt{styles.csv}) where labels correspond to product categories.
    \item Implement cross-validation to assess the model's performance across different subsets of the data.
    \item Fine-tune the model with data augmentation techniques and evaluate using metrics such as accuracy, precision, and recall.
\end{itemize}

\textbf{Prediction:}
The model will predict the category for each fashion item based on the input image.

\section*{Metrics:}
To evaluate the performance of our fashion image classification model, we will use the following metrics:
\begin{itemize}
    \item \textbf{Accuracy:} The proportion of correct predictions (i.e., the number of items correctly classified into their respective categories divided by the total number of predictions).
    \item \textbf{F1-Score:} A metric that balances precision and recall. This is important in situations where we have class imbalances, and we want to account for both false positives and false negatives.
    \begin{equation}
        F1 = 2 \times \frac{\text{precision} \times \text{recall}}{\text{precision} + \text{recall}}
    \end{equation}
    \item \textbf{Precision:} The proportion of true positives among all items classified into a certain category.
    \begin{equation}
        \text{Precision} = \frac{\text{True Positives}}{\text{True Positives} + \text{False Positives}}
    \end{equation}
    \item \textbf{Recall:} The proportion of true positives among all the items that actually belong to a particular category.
    \item \textbf{Confusion Matrix:} A confusion matrix will also be generated to better understand misclassifications between different categories.
\end{itemize}

\section*{Plan and Timeline:}
\begin{tabular}{|l|l|l|}
\hline
\textbf{Phase} & \textbf{Description} & \textbf{Duration} \\
\hline
Data Preprocessing & Resize and normalize images, perform data augmentation, prepare dataset for training. & 1 week \\
\hline
Model Selection and Architecture & Choose between simple CNN or pretrained models like ResNet/EfficientNet. Implement the chosen architecture. & 2 weeks \\
\hline
Model Training & Train the model on the training dataset and validate it on a separate validation set. Implement transfer learning if using pretrained models. & 2 weeks \\
\hline
Evaluation & Evaluate the model using accuracy, F1-score, precision, recall, and confusion matrix. & 1 week \\
\hline
Fine-Tuning and Optimization & Fine-tune model parameters and apply further data augmentation to improve performance. & 1 week \\
\hline
Final Testing and Reporting & Test the model on a separate test set and prepare the final report. & 1 week \\
\hline
\end{tabular}

\section*{Resources:}
\textbf{Hardware:} A high-performance personal computer or GPU instance will be used for training. If necessary, additional cloud-based GPUs can be employed.

\textbf{Software Libraries:}
\begin{itemize}
    \item Python (Programming Language)
    \item TensorFlow or PyTorch (Deep Learning Framework)
    \item Keras (High-level Neural Networks API)
    \item OpenCV or Pillow (Image Preprocessing)
    \item Scikit-learn (For evaluation metrics and model evaluation)
    \item Matplotlib (For data visualization)
\end{itemize}

\section*{Contributions:}
Each team member focuses on the following responsibilities, but we act and work together as a team first and foremost:
\begin{itemize}
    \item \textbf{Team Member 1:} Responsible for data preprocessing and augmentation, setting up the CNN model, and managing training and validation.
    \item \textbf{Team Member 2:} Focus on experimenting with pretrained models (ResNet, EfficientNet) and applying transfer learning techniques to improve performance.
    \item \textbf{Team Member 3:} Handle model evaluation, performance metrics (accuracy, F1-score, precision, recall), and fine-tuning the model for optimal performance.
    \item \textbf{Team Member 4:} Responsible for data preprocessing and augmentation.
\end{itemize}

\end{document}
